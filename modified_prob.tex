%%%%%%%%%%%%%%%%%%%%%%%%%%%%%%%%%%%%%%%%%
% Article Notes
% LaTeX Template
% Version 1.0 (1/10/15)
%
% This template has been downloaded from:
% http://www.LaTeXTemplates.com
%
% Authors:
% Vel (vel@latextemplates.com)
% Christopher Eliot (christopher.eliot@hofstra.edu)
% Anthony Dardis (anthony.dardis@hofstra.edu)
%
% License:
% CC BY-NC-SA 3.0 (http://creativecommons.org/licenses/by-nc-sa/3.0/)
%
%%%%%%%%%%%%%%%%%%%%%%%%%%%%%%%%%%%%%%%%%

%----------------------------------------------------------------------------------------
%	PACKAGES AND OTHER DOCUMENT CONFIGURATIONS
%----------------------------------------------------------------------------------------

\documentclass[
10pt, % Default font size is 10pt, can alternatively be 11pt or 12pt
a4paper, % Alternatively letterpaper for US letter
% twocolumn, % Alternatively onecolumn
% landscape % Alternatively portrait
]{report}

%%%%%%%%%%%%%%%%%%%%%%%%%%%%%%%%%%%%%%%%%
% Article Notes
% Structure Specification File
% Version 1.0 (1/10/15)
%
% This file has been downloaded from:
% http://www.LaTeXTemplates.com
%
% Authors:
% Vel (vel@latextemplates.com)
% Christopher Eliot (christopher.eliot@hofstra.edu)
% Anthony Dardis (anthony.dardis@hofstra.edu)
%
% License:
% CC BY-NC-SA 3.0 (http://creativecommons.org/licenses/by-nc-sa/3.0/)
%
%%%%%%%%%%%%%%%%%%%%%%%%%%%%%%%%%%%%%%%%%

%----------------------------------------------------------------------------------------
%	REQUIRED PACKAGES
%----------------------------------------------------------------------------------------

\usepackage[includeheadfoot,columnsep=2cm, left=1in, right=1in, top=.5in, bottom=.5in]{geometry} % Margins

\usepackage[T1]{fontenc} % For international characters
\usepackage{XCharter} % XCharter as the main font

\usepackage[english]{babel} % Use english by default

% maths packages

\usepackage{amsmath}
\usepackage{xfrac}
\usepackage{multirow}
\usepackage{float}
\usepackage{xcolor}
\usepackage{enumerate}
\usepackage{listings}
\usepackage{amsfonts}

%----------------------------------------------------------------------------------------
%	CUSTOM COMMANDS
%----------------------------------------------------------------------------------------

\newcommand{\articletitle}[1]{\renewcommand{\articletitle}{#1}} % Define a command for storing the article title

\newcommand{\datenotesstarted}[1]{\renewcommand{\datenotesstarted}{#1}} % Define a command to store the date when notes were first made
\newcommand{\docdate}[1]{\renewcommand{\docdate}{#1}} % Define a command to store the date line in the title

\newcommand{\docauthor}[1]{\renewcommand{\docauthor}{#1}} % Define a command for storing the article notes author

% Define a command for the structure of the document title
\newcommand{\printtitle}{
\begin{center}
\topskip0pt
\vspace*{\fill}
\textbf{\Huge{The Notes Project}}

\vspace{2ex}

\textbf{\huge{\articletitle}}

\vspace{1ex}

\Large\docdate \\
\vspace{1ex}
\Large\docauthor \\
\vspace*{\fill}

\end{center}
}

% \renewcommand{\theenumii}{\alph{enumii}.}

\usepackage{fancyhdr}

\newcommand{\ls}[2]{\sum_{#1=1}^{#2}}

\newcommand{\comment}[1]{%
  \text{\phantom{(#1)}} \tag{#1}
}

\newlength\dlf  % Define a new measure, dlf
\newcommand{\alignedbox}[2]{
% Argument #1 = before & if there were no box (lhs)
% Argument #2 = after & if there were no box (rhs)
&  % Alignment sign of the line
{\settowidth\dlf{#1}\addtolength\dlf{\fboxsep+\fboxrule} \hspace{-\dlf} \boxed{#1 #2}}
}
\newcommand{\Perm}[2]{{}^{#1}\!P_{#2}}%
\newcommand{\Comb}[2]{{}^{#1}C_{#2}}%

%----------------------------------------------------------------------------------------
%	STRUCTURE MODIFICATIONS
%----------------------------------------------------------------------------------------

\setlength{\parskip}{3pt} % Slightly increase spacing between paragraphs

% Uncomment to center section titles
\usepackage{sectsty}
\sectionfont{\centering}

% Uncomment for Roman numerals for section numbers
\renewcommand\thesection{\Roman{section}}


\usepackage{pifont}

\usepackage{tikz, pgfplots}
\pgfplotsset{compat=1.18}
\usetikzlibrary{positioning}

\usepackage{gensymb}


% ------------------------
%        PIE CHART
% ------------------------

\newcommand{\pieslice}[6][black!10]{
%%% Usage: \pieslice[color]{total}{start angle}{end angle}{data value}{label}
  % calculate start and end points of arc
  \pgfmathparse{#3/#2*360}
  \let\a\pgfmathresult
  \pgfmathparse{#4/#2*360}
  \let\b\pgfmathresult

  % calculate mid angle of arc
  \pgfmathparse{0.5*\a+0.5*\b}
  \let\midangle\pgfmathresult

  % draw slice
  \draw[fill=#1] (0,0) -- (\a:1) arc (\a:\b:1) -- cycle;

  % outer label
  \node[label=\midangle:{\small#6}] at (\midangle:1) {};

  % inner label
  \pgfmathparse{min((\b-\a-10)/110*(-0.3),0)}
  \let\temp\pgfmathresult
  \pgfmathparse{max(\temp,-0.5) + 0.8}
  \let\innerpos\pgfmathresult
  \pgfmathparse{(\b-\a)/3.6} % convert slice size to percentage
  \let\percentage\pgfmathresult
  \node at (\midangle:\innerpos) {\small\pgfmathprintnumber[fixed,precision=1]{\percentage}\%};
}

\newcommand{\pie}[2][{{"black!10"}}]{
%%% Usage: \pie[{colour palette array}]{{label/value array}}
  % init colour palette
  \pgfmathparse{dim(#1)} % find N of array
  \let\paletteDim\pgfmathresult
  \newcounter{colourIndex}

  % get total for dividing pie into sectors
  \newcounter{total}
  \foreach \val/\name in #2 {
    \addtocounter{total}{\val}
  }

  \newcounter{a}
  \newcounter{b}
  \foreach \val/\name in #2 {
    \setcounter{a}{\value{b}}
    \addtocounter{b}{\val}

    % get colour from palette
    \pgfmathparse{#1[\thecolourIndex]}
    \let\colour\pgfmathresult

    \pieslice[\colour]{\thetotal}{\thea}{\theb}{\val}{\name}

    % increment colour palette
    \stepcounter{colourIndex}
    \ifnum \thecolourIndex=\paletteDim \setcounter{colourIndex}{0}\fi
  }
}

\def\palette{{"blue!60","cyan!50","yellow!50","orange!60","red!60",
    "teal!50","brown!50!black!50","purple!50","lime!50!black!30"}}
    
    
% ---------------------------
%           TALLY
% ---------------------------

\newcount\tmpnum
\def\tallymarks#1{\leavevmode \lower1bp\vbox to9bp{}%
   \tmpnum=#1
   \loop \ifnum\tmpnum<5 \kern1bp \tallynum\tmpnum \else \tallyV \fi
         \advance\tmpnum by-5
         \ifnum\tmpnum>0 \repeat
}
\def\tallynum#1{\bgroup\tmpnum=#1\relax
   \loop \ifnum\tmpnum>0
         \kern1bp \tallyI \kern1bp
         \advance\tmpnum by-1
         \repeat
   \egroup
}
\def\tallyI{\pdfliteral{q .5 w 0 -1 m 0 8 l S Q}}
\def\tallyV{\kern1bp\pdfliteral{q .5 w -1 0 m 9 7 l S Q}\tallynum4\kern1bp }

 % Input the file specifying the document layout and structure

%----------------------------------------------------------------------------------------
%	ARTICLE INFORMATION
%----------------------------------------------------------------------------------------

\articletitle{Probability Theory} % The title of the article

\datenotesstarted{December 15, 2023} % The date when these notes were first made
\docdate{\datenotesstarted; rev. \today} % The date when the notes were lasted updated (automatically the current date)

\docauthor{Keval Mehta, Divyansha Sachdeva} % Your name

%----------------------------------------------------------------------------------------

\begin{document}

\pagestyle{myheadings} % Use custom headers


%----------------------------------------------------------------------------------------
%	PRINT ARTICLE INFORMATION
%----------------------------------------------------------------------------------------

\thispagestyle{plain} % Plain formatting on the first page

\printtitle % Print the title

%----------------------------------------------------------------------------------------
%	ARTICLE NOTES
%----------------------------------------------------------------------------------------

\part{Introduction to Probability Theory}

\subsection*{Basic Definitions}
\begin{description}
\item[Sample Space \((S/\Omega)\)] The set of all possible outcomes.
\item[Event] A subset of the sample space.
\item[Independent Events] Events A \& B defined on \(\Omega\) are independent if they are not affected  by each other.
\item[Mutually Exclusive Events] Events A \& B defined on \(\Omega\) are mutually exclusive if they cannot occur simultaneously.
\item[Exhaustive Events] Events are exhaustive if their union is equal to the sample space.
\end{description}



\subsection*{Types of Probability}
\subsubsection*{Classical Definition of Probability}
    If a random experiment has \(n\) mutually exclusive, equally likely and exhaustive outcomes, and \(m\) of them are favourable to event A, then probability of happening of event A is given by:
    \[
    P(A)=\frac{\text{favourable events}}{\text{total events}}=\frac{m}{n}
    \]
Properties
\begin{enumerate}
\item \(0\leq P(A) \leq 1\)
\item \(P(A')=1-P(A)\)
\end{enumerate}

\subsubsection*{Empirical/Relative Definition of Probability}
 If an experiment is repeated \(n\) times, as \(n\) tends to \(\infty\) and it produces \(m\) outcomes favourable to event A, then probability of happening of event A is given by:
\[
P(A)=\lim_{n\to\infty}\frac{m}{n}
\]
It is useful for unequally likely events, or countably infinite sample spaces.

\subsubsection*{Axiomatic Definition of Probability}
    Let \(\Omega\)/S be a sample space, A be any event defined on sample space, then function P is said to be probability function on probability measure if it satisfies the following axioms.
    \begin{itemize}
    \item \(P(A)\geq 0\)
    \item \(P(\Omega)=1\)
    \item For A \& B, any two mutually exclusive events defined on sample space \(\Omega\), \(P(A\cup B) = P(A)+P(B)\)
    \end{itemize}
    

\subsection*{Various Theorems and Identities}
\paragraph{Addition Theorem:}
\[
P\left(\bigcup_{i=1}^n P(A_i)\right)=\sum_{i=1}^n P(A_i) - \mathop{\sum^n \sum^n}\limits_{0 \leq i < j=1} P(A_i \cap A_j) + \mathop{\sum^n \sum^n \sum^n}\limits_{0 \leq i < j < k=1} P(A_i \cap A_j \cap A_k) \dots \]


\paragraph{Conditional Probability:}

If A \& B are two events, probability of happening of event A if B has already happened is given by:

\[
P(A|B) = \frac{P(A\cap B)}{P(B)}
\]
Similarly, 
\[
P(B|A) = \frac{P(A\cap B)}{P(A)}
\]

\paragraph{Multiplication Theorem:}

\begin{align*}
    P(A\cap B) &= P(A|B) \cdot P(B) \\
               &= P(B|A) \cdot P(A)
\end{align*}

\paragraph{Bayes' Theorem}
If \(A_1, A_2, \dots A_n\) are \(n\) events defined on sample space such that \(\bigcup_{i=1}^n P(A_i)=1\) and \\ \(\bigcap_{i=1}^n P(A_i)=\phi\), B be an event defined on same sample space such that \(B\subseteq \cup A_i, P(B)\neq 0\):
\[
P(A_i| B) = \frac{P(B|A_i)\cdot P(A_i)}{\sum P(B|A_i)\cdot P(A_i)}
\]

\subsection*{More Definitions}
\begin{description}
  \item[Random Variable]
    A real-valued function defined on sample space.
  \item[Probability Distribution]
    The set of pairs of values of a random variable and the probability of those values.
  \item[Probability Mass Function]
    Let \(X\) be a discrete random variable and \(P(X=x)\) be a function defined on X. \(P(X=x)\) is said to be its p.m.f. if it satisfies the following conditions: \\
    1) \(P(x)\geq 0\) \\
    2) \(\sum P(x) = 1\)
  \item[Probability Density Function]
    Let \(X\) be a continuous random variable and \(f(X=x)\) be a function defined on X. \(f(X=x)\) is said to be its p.d.f. if it satisfies the following conditions: \\
    1) \(f(x)\geq 0\) \\
    2) \(\int f(x) = 1\)
  \item[Cumulative Distribution Function]
    Let \(X\) be a random variable and \(P(X=x)\) be its p.m.f. Distribution function is given by:
    \[
    f(x)= 
        \begin{cases}
            \sum_{i=1}^x P(X=x),        & \text{for discrete r.v.} \\ \\
            \int_{-\infty}^x f(x)dx,    & \text{for continuous r.v.}
        \end{cases}
    \]
  \item[Expectation of a Random Variable]
    Let \(X\) be a r.v. with p.m.f. \(P(X=x)\) (for discrete) and \(f(x)\) (for continuous). Then, expectation is denoted by \(E(X)\) and is given by:
    \[
    E(x)= 
        \begin{cases}
            \sum x\cdot P(X=x),    & \text{for discrete r.v.} \\ \\
            \int x\cdot f(x)dx,    & \text{for continuous r.v.}
        \end{cases}
    \]
\item[Variance of a Random Variable]
    Let \(X\) be a r.v. with expectation \(E(X)\). Then, variance is denoted by \(Var(X)\) and is given by:
    \[
    Var(X) = E(X^2)-[E(X)]^2
    \]
\end{description}

\subsection*{Moments}
\subsubsection*{Definitions:}
\begin{description}
  \item[Central Moments]
    \[
    \mu_r = \frac{\sum(x_i-\mu)^r}{n}
    \]
  \item[Raw Moments]
    \[
    \mu'_r = \frac{\sum x_i^r}{n}
    \]
  \item[Arbitrary Moments]
    \[
    \mu_{r_A} = \frac{\sum(x_i-A)^r}{n}
    \]
    
\end{description}

\subsubsection*{Functions:}
\begin{description}
  \item[Moment Generating Function]
    \[
    M_x(t) = E(e^{tx})
    \]
  \item[Cumulant Generating Function]
    \[
    K_x(t) = log(M_x(t)) = log(E(e^{tx}))
    \]
  \item[Characteristic Function ]
    \[
    \phi_x(t) = E(e^{itx})
    \]
    
\end{description}

\part{Discrete Probability Distributions}

\section*{Uniform Distribution:}
\subsection*{Definition:}
Let X be a discrete r.v. It follows uniform distribution if its p.m.f is:
\[P(X=x)=\frac{1}{n};\; x=1, 2\dots n\] 
It is denoted by \(X \sim D(n).\)

\subsection*{Applications:}
1) Tossing a coin, getting heads or tails. \\
2) Selection of a student from a class.

\subsection*{Expectation and Variance:}
\begin{align*}
    E(X) &= \sum x \cdot P(x) \\
         &= \sum_{x=1}^n x\cdot\frac{1}{n} \\
         &= \frac{1}{n} \cdot \frac{n(n+1)}{2} \\
    E(X) &= \frac{(n+1)}{2} \\ \\
    E(X^2) &= \sum x^2 \cdot P(x) \\
           &= \sum_{x=1}^n x^2\cdot\frac{1}{n} \\
           &= \frac{1}{n} \cdot \frac{n(n+1)(2n+1)}{6} \\
    E(X^2) &= \frac{(n+1)(2n+1)}{6} \\ \\
    Var(x) &= E(X^2) - [E(X)]^2 \\
           &= \frac{(n+1)(2n+1)}{6} - \left[\frac{(n+1)}{2}\right]^2 \\
           &= \frac{2n^2+3n+2}{6}-\frac{n^2+2n+1}{4} \\
           &= \frac{4n^2+6n+4-3n^2-6n-3}{12} \\
    Var(X) &= \frac{n^2-1}{12} = \frac{(n+1)(n-1)}{12}
\end{align*}
\newpage

\subsection*{Moment Generating Function:}
Given \(X \sim D(n), f(x)=\dfrac{1}{n}:\)
\begin{align*}
    M_x(t) &= E(e^{tx}) \\
           &= \sum e^{tx} \cdot P(x) \\
           &= \sum_{x=1}^n e^{tx} \cdot \frac{1}{n} \\
           &= \frac{1}{n} \sum_{x=1}^n e^{tx} \\
    M_x(t) &= \frac{e^t(1-e^{nt})}{n(1-e^t)}
\end{align*}


\newpage

\section*{Bernoulli Distribution:}
\subsection*{Definition:}
Let X be a discrete r.v. It follows uniform distribution if its p.m.f is:
\[
    P(x)= 
\begin{cases}
    p^x q^{1-x},    & \text{if } x = 0,1\\
    0,              & \text{otherwise}
\end{cases}
\]
It is denoted by \(X \sim B(p).\)

\subsection*{Applications:}
1) Tossing a coin, getting heads or tails. \\
2) Rolling a die, getting odd or even. \\
3) Picking a card, getting red or black. \\
4) Selecting item, defective or not defective.

\subsection*{Expectation and Variance:}
\begin{align*}
    E(X) &= \sum x \cdot P(x) \\
         &= \sum_{x=0}^1 x\cdot p^x q^{1-x} \\
         &= 0\cdot p^0 \cdot q + 1\cdot p \cdot q^0 \\
    E(X) &= p \\ \\
    E(X^2) &= \sum x^2 \cdot P(x) \\
           &= \sum_{x=0}^1 x^2\cdot p^x q^{1-x} \\
           &= 0\cdot p^0 \cdot q + 1\cdot p \cdot q^0 \\
    E(X^2) &= p \\ \\
    Var(x) &= E(X^2) - [E(X)]^2 \\
           &= p-p^2 \\
           &= p(1-p) \\
    Var(X) &= pq
\end{align*}

\subsection*{Moment Generating Function:}
Given \(X \sim B(n), f(x)=p^x q^{1-x}:\)
\begin{align*}
    M_x(t) &= E(e^{tx}) \\
           &= \sum e^{tx} \cdot P(x) \\
           &= \sum_{x=0}^1 e^{tx} \cdot p^x q^{1-x} \\
           &= e^{0t} \cdot q + e^{1t} \cdot p\\
    M_x(t) &= q+pe^t
\end{align*}
\newpage

\section*{Binomial Distribution:}
\subsection*{Definition:}
Let X be a discrete r.v. It follows uniform distribution if its p.m.f is:
\[
    P(x)= 
\begin{cases}
    \binom{n}{x}\;p^x q^{n-x},        & \text{if } x = 0,1\dots n\\
    0,                              & \text{otherwise}
\end{cases}
\]
It is denoted by \(X \sim B(n, p).\)

\subsection*{Applications:}
1) Tossing multiple coins, getting heads or tails certain number of times. \\
2) Repeating any Bernoulli event \(n\) number of times.

\subsection*{Expectation and Variance:}
\begin{align*}
    E(X) &= \sum x \cdot P(x) \\
         &= \sum_{x=0}^n x\cdot \binom{n}{x} p^x q^{1-x} \\
         &= 0 + 1\cdot \binom{n}{1} \cdot p^1 \cdot q^{n-1} + 2\cdot \binom{n}{2} \cdot p^2 \cdot q^{n-2} + \dots + n\cdot \Comb{n}{n} \cdot p^n \cdot q^{n-n} \\
         &= npq^{n-1} + n(n-1)p^2 q^{n-2} + n(n-1)(n-2) p^3 q^{n-3} + \dots + np^n \\
         &= np[q^{n-1}+(n-1)pq^{n-2}+\dots+p^{n-1}] \\
\comment{\(p+q=1\)}         &= np(p+q)^{n-1} \\
    E(X) &= np \\ \\
    E(X^2) &= \sum x^2 \cdot P(x) \\
           &= \sum_{x=0}^n x+x(x-1) \cdot \binom{n}{x} p^x q^{n-x} \\
           &= \sum_{x=0}^n x \cdot \binom{n}{x} p^x q^{n-x}\sum_{x=0}^n x(x-1) \cdot \binom{n}{x} p^x q^{n-x} \\
           &= np + \sum_{x=0}^n x(x-1) \cdot \binom{n}{x} p^x q^{n-x} \\
           &= np + (0 + 0 + n(n-1)p^2 q^{n-2} + n(n-1)(n-2) p^3 q^{n-3} + \dots + np^n)\\
           &= np + n(n-1)p(p+q)^{n-2} \\
    E(X^2) &= np + n^2p^2 - np^2 \\ \\
    Var(X) &= E(X^2) - [E(X)]^2 \\
           &= np + n^2p^2 - np^2 - n^2p^2 \\
           &= np - np^2 \\
           &= np(1-p) \\
    Var(X) &= npq
\end{align*}
Note:
If \(X_1, X_2, X_3, \dots, X_k\) are independent B\((n, p)\) then \(\sum_{i=1}^k X_i \sim B(\sum_{i=1}^k n_i, p)\).

\subsection*{Assumptions:}
1) The number of trials \(n\) is fixed and finite. \\
2) The probability of success \(p\) is the same for every trial. \\
3) \(p+q=1\), where \(q\) is the probability of failure. \\
4) \(p\) is independent for every trial.

\subsection*{Moment Generating Function:}
Given \(X \sim B(n, p), f(x)=\binom{n}{x} p^x q^{n-x}:\)
\begin{align*}
    M_x(t) &= E(e^{tx}) \\
           &= \sum e^{tx} \cdot P(x) \\
           &= \sum_{x=0}^n e^{tx} \cdot \binom{n}{x} \; p^x q^{n-x} \\
           &= \sum_{x=0}^n \; \binom{n}{x} \; (pe^t)^x \; q^{n-x}\\
    M_x(t) &= (q+pe^t)^n
\end{align*}

\newpage

\section*{Poisson Distribution:}

Let X be a discrete r.v. It follows uniform distribution if its p.m.f is:
\[
    P(x)= 
\begin{cases}
    \dfrac{e^{-\lambda} \lambda^x}{x!}        & \text{if } x = 0,1\dots\\
    0,                              & \text{otherwise}
\end{cases}
\]
It is denoted by \(X \sim P(\lambda).\)

\subsection*{Expectation and Variance:}
\begin{align*}
    E(X) &= \sum_{n=0}^\infty x \cdot P(x) \\
         &= \sum_{n=0}^\infty \frac{x\;e^{-\lambda}\;\lambda^x}{x!} \\
         &= e^{-\lambda} \sum_{n=0}^\infty \frac{x\; \lambda^x}{x!} \\
         &= e^{-\lambda} \left(0+\frac{\lambda}{1!}+\frac{\lambda^2}{2!}+ \dots \right) \\
         &= e^{-\lambda} \cdot \lambda \left(1+ \frac{\lambda}{2!} \dots\right) \\
         &= e^{-\lambda} \cdot \lambda \cdot e^{\lambda} && \sum_{n=0}^\infty \frac{\lambda^x}{x!} = e^{\lambda} \\
    E(X) &= \lambda \\ \\
    E(X^2) &= \sum x^2 \cdot P(x) \\
           &= \sum_{n=0}^\infty \frac{x^2\;e^{-\lambda}\;\lambda^x}{x!} \\
           &= e^{-\lambda} \sum_{n=0}^\infty \frac{x+x(x-1)\; \lambda^x}{x!} \\
           &= e^{-\lambda} \left[ \sum_{n=0}^\infty \frac{x\; \lambda^x}{x!} + \sum_{n=0}^\infty \frac{x(x-1)\; \lambda^x}{x!}\right] \\
           &= e^{-\lambda} \left[\lambda e^{\lambda}+ \left(0+0+\frac{2\lambda^2}{2!}+\frac{6\lambda^3}{3!}+ \dots \right)\right] \\
           &= e^{-\lambda} \left[\lambda e^{\lambda}+ \lambda^2\left(1+\lambda+ \frac{\lambda^2}{2!} + \frac{\lambda^3}{3!}\dots \right)\right] \\
           &= e^{-\lambda} \left[\lambda e^{\lambda}+ \lambda^2 e^{\lambda}\right] \\
    E(X^2) &= \lambda+\lambda^2 \\ \\
    Var(X) &= E(X^2) - [E(X)]^2 \\
           &= \lambda+\lambda^2-\lambda^2 \\
    Var(X) &= \lambda
\end{align*}

\subsection*{Assumptions}
1) Mean = Variance = \(\lambda\). \\
2) \(\sigma=\sqrt{\lambda}\). \\
3) If \(X_1, X_2, X_3, \dots, X_k\) are independent P\((\lambda_i)\) then \(\sum_{i=1}^k X_i \sim P(\sum_{i=1}^k \lambda_i)\).

\subsection*{Applications:}
1) Probability of rain in many summers. \\
2) Probability of a misprint in a page across a library. \\
3) Probability of an accident in a large parking lot.

\subsection*{Moment Generating Function:}
Given \(X \sim P(\lambda), f(x)=\dfrac{e^{-\lambda} \lambda^x}{x!}:\)
\begin{align*}
    M_x(t) &= E(e^{tx}) \\
           &= \sum e^{tx} \cdot P(x) \\
           &= \sum_{x=0}^\infty e^{tx} \cdot \frac{e^{-\lambda} \lambda^x}{x!} \\
           &= e^{-\lambda} \sum_{x=0}^\infty \frac{e^{tx} \lambda^x}{x!} \\
           &= e^{-\lambda} \sum_{x=0}^\infty \frac{(\lambda e^t)^x}{x!} \\
           &= e^{-\lambda} \cdot e^{\lambda e^t} && {\sum_{x=0}^\infty \frac{\lambda^x}{x!} = e^\lambda} \\
    M_x(t) &= e^{\lambda(e^t-1)}
\end{align*}

\newpage
\section*{Geometric Distribution:}
\subsection*{Definition:}
Let X be a discrete r.v. It follows uniform distribution if its p.m.f is:
\paragraph{Type 1:}
\[
    P(x)= 
\begin{cases}
    p q^{x-1},    & \text{if } x = 1, 2, \dots\\
    0,              & \text{otherwise}
\end{cases}
\]
It is denoted by \(X \sim G(p).\) In this case, \(x\) is the number of trials.\\

\paragraph{Expectation and Variance:}
\begin{align*}
    E(X) &= \sum x \cdot P(x) \\
         &= \sum x \cdot pq^{x-1} \\
         &= p\sum x q^{x-1} \\
         &= p \left[ 0+1+2q+3q^2\dots \right] \\
         &= p \cdot \frac{1}{p^2} \\
    E(X) &= \frac{1}{p} \\ \\
    E(X^2) &= \sum x^2 \cdot P(x) \\
           &= \sum [x+x(x-1)] \cdot pq^{x-1} \\
           &= \sum x \cdot pq^{x-1} + \sum x(x-1) \cdot pq^{x-1} \\
           &= \frac{1}{p} + p\sum x(x-1) \cdot q^{x-1} \\
           &= \frac{1}{p} + p(0 + 0 + 2q + 6q^2 + 12q^3 \dots)\\
           &= \frac{1}{p} + 2pq(1+3q+6q^2 \dots) \\
           &= \frac{1}{p} + p\cdot q\cdot \frac{1}{p^3} \\
           &= \frac{p+2q}{p^2} \\
           &= \frac{p+q+q}{p^2} \\
    E(X^2) &= \frac{1+q}{p^2} \\ \\
    Var(X) &= E(X^2) - [E(X)]^2 \\
           &= \frac{q+1}{p^2} - \frac{1}{p^2} \\
           &= \frac{q}{p^2} + \frac{1}{p^2} - \frac{1}{p^2} \\
    Var(X) &= \frac{q}{p^2}
\end{align*}

\paragraph{Type 2:}
\[
    P(x)= 
\begin{cases}
    p q^x,    & \text{if } x = 0, 1, 2, \dots\\
    0,              & \text{otherwise}
\end{cases}
\]
It is denoted by \(X \sim G(p).\) In this case, \(x\) is the number of failures.\\

\paragraph{Expectation and Variance:}
\begin{align*}
    E(X) &= \sum x \cdot P(x) \\
         &= \sum x \cdot pq^x \\
         &= p\sum x q^x \\
         &= p [q+2q^2+3q^3\dots] \\
         &= pq \left[1+2q+3q^2\dots \right] \\
         &= pq \cdot \frac{1}{p^2} \\
    E(X) &= \frac{q}{p} \\ \\
    E(X^2) &= \sum x^2 \cdot P(x) \\
           &= \sum [x+x(x-1)] \cdot pq^x \\
           &= \sum x \cdot pq^x + \sum x(x-1) \cdot pq^x \\
           &= \frac{q}{p} + p\sum x(x-1) \cdot q^x \\
           &= \frac{q}{p} + p(0 + 0 + 2q^2 + 6q^3 + 12q^4 \dots)\\
           &= \frac{q}{p} + 2pq^2(1+3q+6q^2 \dots) \\
           &= \frac{q}{p} + 2p\cdot q^2\cdot \frac{1}{p^3} \\
    E(X^2) &= \frac{q}{p} + \frac{2p^2}{q^2} \\ \\
    Var(X) &= E(X^2) - [E(X)]^2 \\
           &= \frac{q}{p} + \frac{2p^2}{q^2} - \frac{q^2}{p^2} \\
           &= \frac{q}{p} + \frac{q^2}{p^2} \\
           &= \frac{pq+q^2}{p^2} \\
           &= \frac{q(p+q)}{p^2} \\
    Var(X) &= \frac{q}{p^2}
\end{align*}

\paragraph{Moment Generating Function:}
Given \(X \sim G(p), f(x)=pq^x:\)
\begin{align*}
    M_x(t) &= E(e^{tx}) \\
           &= \sum e^{tx} \cdot P(x) \\
           &= \sum_{x=0}^\infty e^{tx} \cdot pq^x \\
           &= p \sum_{x=0}^\infty (qe^t)^x \\
    M_x(t) &= \dfrac{p}{1-qe^t}
\end{align*}

\newpage

\section*{Negative Binomial Distribution:}
\subsection*{Definition:}
Let X be a discrete r.v. It follows uniform distribution if its p.m.f is:
\[
    P(x)= 
\begin{cases}
    \binom{k+r-1}{r-1} p^r q^x,    & \text{if } x = 0, 1, 2, \dots\\
    0,              & \text{otherwise}
\end{cases}
\]
It is denoted by \(X \sim G(p).\) 

\part{Continuous Probability Distributions}

\subsection*{Continuous Probability Distributions}
\subsubsection*{Rectangular Distribution:}
\paragraph*{Definition:}
If c.r.v \(X\sim U(a, b)\), then its p.d.f is:
\[
    f(x)= 
\begin{cases}
    \dfrac{1}{b-a},        & \text{if } a\leq x \leq b\\ \\
    0,                    & \text{otherwise}
\end{cases}
\]

\paragraph*{Expectation, Median, Mode and Variance:}
\begin{align*}
    E(X) &= \int_a^b x \cdot f(x) dx \\
         &= \int_a^b \frac{x}{b-a} dx \\
         &= \frac{x^2}{2(b-a)} \Biggr|_a^b \\
         &= \frac{b^2-a^2}{2(b-a)} \\
         &= \frac{(b+a)(b-a)}{2(b-a)} \\
    E(X) &= \frac{b+a}{2} \\ \\
    E(X^2) &= \int_a^b x^2 \cdot f(x) dx \\
           &= \int_a^b \frac{x^2}{b-a} dx \\
           &= \frac{x^3}{3(b-a)} \Biggr|_a^b \\
           &= \frac{b^3-a^3}{2(b-a)} \\
           &= \frac{(b^2+ba+a^2)(b-a)}{3(b-a)} \\
    E(X^2) &= \frac{b^2+ba+a^2}{3}  \\ \\
    Var(x) &= E(X^2) - [E(X)]^2 \\
           &= \frac{b^2+ba+a^2}{3} - \left[\frac{b+a}{2}\right]^2 \\
           &= \frac{4a^2+4ab+4b^2-3a^2-6ab-3b^2}{12} \\
           &= \frac{b^2-2ab+a^2}{12} \\
    Var(X) &= \frac{(b-a)^2}{12}
\end{align*}

Standard Deviation \(\sigma\):
\begin{align*}
    \sigma &= \sqrt{Var(x)} \\
    \sigma &= \dfrac{(b-a)}{\sqrt{12}}
\end{align*}

To find median \(M\):
\begin{align*}
    \int_a^M f(x) dx &= \frac{1}{2} \\
    \frac{M-a}{b-a} &= \frac{1}{2} \\
    2M-2a &= b-a \\
    M &= \frac{b+a}{2}
\end{align*}


Mode of Rectangular Distribution is every \(x\) such that \(a\leq x \leq b\) as:
\[
\frac{d}{dx}\;\frac{1}{b-a}=0
\]
Therefore, all points are its maxima and minima.

\paragraph*{Moment Generating Function:}

\begin{align*}
    M_x(t) &= E(e^{tx}) \\
           &= \int_a^b e^{tx} \cdot f(x) dx \\
           &= \frac{1}{b-a} \; \frac{e^{tx}}{t} \Biggr|_a^b \\
           &= \frac{1}{t(b-a)}\cdot (e^{bt}-e^{at}) \\
    M_x(t) &= \frac{e^{bt}-e^{at}}{t(b-a)}
\end{align*}

First Raw Moment:
\begin{align*}
    \mu'_r &= \int_a^b x^r f(x) dx \\
    \mu'_r &= \frac{1}{b-a}\left[\frac{b^{r+1}-a^{r+1}}{r+1}\right]
\end{align*}

\paragraph*{Cumulant Generating Function:}

\[
K_x(t)= log\left[\frac{e^{bt}-e^{at}}{t(b-a)}\right]
\]

C.G.F for \(t=1\):
\[
    K_x(1)= log\left[\frac{e^{b}-e^{a}}{b-a}\right]
\]

\paragraph*{Characteristic Function:}
\begin{align*}
    \phi_x(t) &= E(e^{itx}) \\
    \phi _x(t) &= \frac{e^{ibt}-e^{iat}}{it(b-a)}
\end{align*}
    
\subsubsection*{Triangular Distribution:}
\paragraph*{Definition:}
If c.r.v \(X\sim T(a, b)\) with mode \(c\), then its p.d.f is:
\[
    f(x)= 
\begin{cases}
    \dfrac{2(x-a)}{(b-a)(c-a)},        & \text{if } a\leq x \leq c\\ \\
    \dfrac{2(b-x)}{(b-a)(b-c)},        & \text{if } c\leq x \leq b\\ \\
    0,                                & \text{otherwise}
\end{cases}
\]

\[
f(c)=\frac{2}{b-a}
\]

\paragraph*{Moment Generating Function:}
Moment Generating Function of \(X\sim T(a, b)\) with mode \(c\) is:
\[
M_x(t)=\frac{2}{t^2}\left[\frac{e^{at}}{(a-b)(a-c)}\frac{e^{ct}}{(c-a)(c-b)}\frac{e^{bt}}{(b-a)(b-c)}\right]
\]

If \(X\) \& \(Y\) are i.i.d. \(U(-a, a)\), then addition of \(X\) \& \(Y\), i.e. \(X+Y\sim T(-2a, 2a)\), with mode \(0\). \\
Additionally, \(X-Y\sim T(-2a, 2a)\), with mode \(0\).

\paragraph*{Properties:}
1) \(- \infty < a < b < \infty, c \in [a, b]\) \\
2) if \(C < E(X)\), \text{distribution is positively skewed.} \\
    if \(C > E(X)\), \text{distribution is negatively skewed.} \\
    if \(C=E(X)\), \text{distribution is symmetric.}


\newpage

\subsubsection*{Gamma Distribution:}
\paragraph*{One Parameter:}
If c.r.v \(X\sim \gamma(\lambda)\), then its p.d.f is:
\[
    f(x)= 
\begin{cases}
    \dfrac{e^{-x} x^{\lambda -1}}{\Gamma \lambda}, & \text{if } 0<  x < \infty,\;\lambda >0\\ \\
    0,                                     & \text{otherwise}
\end{cases}
\]

Properties:
\[
\Gamma n = \int_0^\infty e^{-x} x^{n -1} dx
\]
\[
\Gamma (n+1) = n \Gamma n
\]
\[
\Gamma n = (n-1)!
\]

\paragraph{Expectation and Variance:}
\begin{align*}
    E(X) &= \int_0^\infty x \cdot f(x) dx \\
         &= \frac{x\;e^{-x} x^{\lambda -1}}{\Gamma \lambda} \\
         &= \frac{e^{-x} x^{\lambda}}{\Gamma \lambda} \\
         &= \frac{\Gamma (\lambda+1)}{\Gamma \lambda} \\
         &= \frac{\lambda\;\Gamma \lambda}{\Gamma \lambda} \\
    E(X) &= \lambda \\ \\
    E(X^2) &= \int_0^\infty x^2 \cdot f(x) dx \\
           &= \frac{x^2\;e^{-x} x^{\lambda -1}}{\Gamma \lambda} \\
           &= \frac{e^{-x} x^{\lambda+1}}{\Gamma \lambda} \\
           &= \frac{\Gamma (\lambda+2)}{\Gamma \lambda} \\
           &= \frac{\lambda(\lambda+1)\Gamma \lambda}{\Gamma \lambda} \\
    E(X^2) &= \lambda(\lambda+1) = \lambda^2+\lambda  \\ \\
    Var(x) &= E(X^2) - [E(X)]^2 \\
           &= (\lambda^2+\lambda) - \lambda^2 \\
    Var(X) &= \lambda
\end{align*}

\paragraph{Moment Generating Function:}
\[
M_x(t)=(1-t)^{-\lambda}
\]

\paragraph{Cumulant Generating Function:}
The value of the \(n^{\text{th}}\) cumulant is \(\lambda(n-1)!\).

\paragraph{Raw Moments:}
The value of the \(r^{\text{th}}\) raw moment is
\begin{align*}
    \mu'_r &= \int_0^\infty x^r\:f(x)dx \\
           &= \int_0^\infty x^r \cdot \frac{e^{-x} x^{\lambda -1}}{\Gamma \lambda} \\
           &= \frac{e^{-x} x^{\lambda + r -1}}{\Gamma \lambda} \\
           &= \frac{\Gamma (\lambda+r)}{\Gamma \lambda} \\
    \mu'_r &= \frac{\Gamma (\lambda+r)}{\Gamma \lambda} \\
    \mu'_r &= \Pi_{i=0}^{n-1}(\lambda+i)
\end{align*}

\paragraph{Coefficients of Skewness and Kurtosis:}
The coefficients of skewness and kurtosis for the gamma distribution are:
\begin{align*}
    \beta_1 &= \frac{\mu_3^2}{\mu_2^3} \\
            &= \frac{(2\lambda)^2}{\lambda^3} \\
    \beta_1 &= \frac{4}{\lambda} \\ \\
    \gamma_1 &= \sqrt{\beta_1} \\
             &= \sqrt{\frac{4}{\lambda}} \\
    \gamma_1 &= \frac{2}{\sqrt{\lambda}} \\ \\
    \beta_2 &= \frac{\mu_4}{\mu_2^2} \\
            &= \frac{6\lambda}{\lambda^2} \\
    \beta_2 &= \frac{6}{\lambda} \\ \\
    \gamma_2 &= \beta_2 - 3 \\
    \gamma_2 &= \frac{6}{\lambda} -3 \\
\end{align*}

Note:
If \(X_1, X_2, X_3, \dots, X_n\) are independent \(\gamma(\lambda_i)\) then \(\sum_{i=1}^n X_i \sim \gamma(\sum_{i=1}^n \lambda_i)\).
\\ \\
\paragraph*{Two Parameter:}
If c.r.v \(X\sim G(\lambda, a)\), then its p.d.f is:
\[
    f(x)= 
\begin{cases}
    \dfrac{a^\lambda e^{-ax} x^{\lambda -1}}{\Gamma \lambda}, & \text{if } 0<  x < \infty,\;\lambda >0, a>0\\ \\
    0,                                     & \text{otherwise}
\end{cases}
\]

\paragraph{Expectation and Variance:}
\begin{align*}
    E(X) &= \int_0^\infty x \cdot f(x) dx \\
         &= \int_0^\infty\frac{a^\lambda e^{-ax} x^{\lambda -1}}{\Gamma \lambda} dx \\
         &= \int_0^\infty\frac{a^\lambda e^{-u} u^{\lambda+1}}{a^{\lambda+1}\;\Gamma \lambda} du && u=ax,\;dx=\frac{du}{a},\; x=\frac{u}{a}\\
         &= \int_0^\infty\frac{e^{-u} u^{\lambda+1}}{a^2} du \\
         &= \frac{\lambda\;\Gamma \lambda}{a\;\Gamma \lambda} \\
    E(X) &= \frac{\lambda}{a} \\ \\
    E(X^2) &= \int_0^\infty x^2 \cdot f(x) dx \\
         &= \int_0^\infty\frac{a^\lambda e^{-ax} x^{\lambda +1}}{\Gamma \lambda} dx \\
         &= \int_0^\infty\frac{a^\lambda e^{-u} u^{\lambda+1}}{a^{\lambda+2}\;\Gamma \lambda} du && u=ax,\;dx=\frac{du}{a},\; x=\frac{u}{a}\\
         &= \int_0^\infty\frac{e^{-u} u^{\lambda+1}}{a^2\;\Gamma \lambda} du \\
         &= \frac{\Gamma (\lambda+2)}{a^2\;\Gamma \lambda} \\
         &= \frac{\lambda(\lambda+1)\;\Gamma \lambda}{a^2\;\Gamma \lambda} \\
    E(X^2) &= \frac{\lambda(\lambda+1)}{a^2} \\ \\
    Var(x) &= E(X^2) - [E(X)]^2 \\
           &= \frac{\lambda(\lambda+1)}{a^2} - \frac{\lambda}{a}^2 \\
           &= \frac{\lambda^2+\lambda}{a^2} - \frac{\lambda}{a}^2 \\
    Var(X) &= \frac{\lambda}{a^2}
\end{align*}

\paragraph{Moment Generating Function:}
\[
M_x(t)=(1-\sfrac{t}{a})^{-\lambda}
\]

\paragraph{Cumulant Generating Function:}
The value of the \(n^{\text{th}}\) cumulant is \(\frac{\lambda(n-1)!}{a^n}\).

\paragraph{Raw Moments:}
The value of the \(r^{\text{th}}\) raw moment is
\begin{align*}
    \mu'_r &= \int_0^\infty x^r\:f(x)dx \\
           &= \int_0^\infty x^r \cdot a^{\lambda} \cdot \frac{e^{-ax} x^{\lambda -1}}{\Gamma \lambda} \\
           &= \int_0^\infty \frac{a^{\lambda} \cdot e^{-ax} x^{\lambda + r -1}}{\Gamma \lambda} \\
           &= \frac{\Gamma (\lambda+r)}{\Gamma \lambda} \\
    \mu'_r &= \frac{\Gamma (\lambda+r)}{\Gamma \lambda} \\
    \mu'_r &= \Pi_{i=0}^{n-1}(\lambda+i)
\end{align*}

\paragraph{Coefficients of Skewness and Kurtosis:}
The coefficients of skewness and kurtosis for the gamma distribution are:
\begin{align*}
    \beta_1 &= \frac{\mu_3^2}{\mu_2^3} \\
            &= \dfrac{\sfrac{(2\lambda)^2}{\lambda^3}}{\sfrac{a^6}{a^6}} \\
    \beta_1 &= \frac{4}{\lambda} \\ \\
    \gamma_1 &= \sqrt{\beta_1} \\
             &= \sqrt{\frac{4}{\lambda}} \\
    \gamma_1 &= \frac{2}{\sqrt{\lambda}} \\ \\
    \beta_2 &= \frac{\mu_4}{\mu_2^2} \\
            &= \dfrac{\sfrac{6\lambda}{\lambda^2}}{\sfrac{a^4}{a^4}} \\
    \beta_2 &= \frac{6}{\lambda} \\ \\
    \gamma_2 &= \beta_2 - 3 \\
    \gamma_2 &= \frac{6}{\lambda} -3 \\
\end{align*}

Note:
If \(X_1, X_2, X_3, \dots, X_n\) are independent \(\gamma(\lambda_i, a)\) then \(\sum_{i=1}^n X_i \sim \gamma(\sum_{i=1}^n \lambda_i, a)\).

\newpage

\subsubsection*{Beta Distribution:}

The p.d.f of the general beta function is given by:
\[
\int_a^b \dfrac{(x-a)^{m-1}(b-x)^{n-1}}{(b-a)^{m+n-1}}, a<x<b
\]

Types of \(\beta\) distribution: \\
If \(a\neq 0\) \& \(b\neq 0\) or \(a\neq 0\) \& \(b \neq \infty\), it is called an incomplete beta distribution. \\
If \(a=0\) \& \(b=1\), it is the first type of beta distribution. \\
If \(a=0\) \& \(b=\infty\), it is the second type of beta distribution. \\

\paragraph*{Type 1(\(\beta_1\))}
If c.r.v \(X\sim \beta_1(m, n)\), then its p.d.f is:
\[
    f(x)= 
\begin{cases}
    \dfrac{x^{m-1}(1-x)^{n-1}}{\beta(m, n)}, & \text{if } 0\leq x \leq 1;m, n>0\\ \\
    0,                                     & \text{otherwise}
\end{cases}
\]

Properties:
\[
\beta(m, n) = \int_0^1 x^{m-1} (1-x)^{n-1} dx
\]
\[
\beta(m, n) = \frac{\Gamma m \; \Gamma n}{\Gamma (m+n)}
\]

\paragraph{Expectation and Variance:}
\begin{align*}
    E(X) &= \int_0^1 x \cdot f(x) dx \\
         &= \int_0^1 \frac{x\;x^{m-1}(1-x)^{n-1}}{\beta(m, n)} \\
         &= \int_0^1 \frac{x^{m}(1-x)^{n-1}}{\beta(m, n)} \\
         &= \frac{\beta(m+1, n)}{\beta(m, n)} \\
         &= \frac{\Gamma (m+1) \; \Gamma n}{\Gamma (m+n+1)} \frac{\Gamma (m+n)}{\Gamma m \; \Gamma n} \\
         &= \frac{m \Gamma m \; \Gamma n}{(m+n)\Gamma (m+n)} \frac{\Gamma (m+n)}{\Gamma m \; \Gamma n} \\
    E(X) &= \frac{m}{m+n} \\ \\
    E(X^2) &= \int_0^1 x^2 \cdot f(x) dx \\
         &= \int_0^1 \frac{x^2\;x^{m-1}(1-x)^{n-1}}{\beta(m, n)} \\
         &= \int_0^1 \frac{x^{m+1}(1-x)^{n-1}}{\beta(m, n)} \\
         &= \frac{\beta(m+2, n)}{\beta(m, n)} \\
         &= \frac{\Gamma (m+2) \; \Gamma n}{\Gamma (m+n+2)} \frac{\Gamma (m+n)}{\Gamma m \; \Gamma n} \\
         &= \frac{(m+1)m \Gamma m \; \Gamma n}{(m+n+1)(m+n)\Gamma (m+n)} \frac{\Gamma (m+n)}{\Gamma m \; \Gamma n} \\
    E(X^2) &= \frac{m(m+1)}{(m+n)(m+n+1)}  \\ \\
    Var(x) &= E(X^2) - [E(X)]^2 \\
           &= \frac{m(m+1)}{(m+n)(m+n+1)} - \frac{m}{m+n} \\
           &= \frac{m}{m+n}\left[\frac{m+1}{m+n+1}-\frac{m}{m+n}\right] \\
           &= \frac{m}{m+n}\left[\frac{(m+1)(m+n)-m(m+n+1)}{(m+n)(m+n+1)}\right] \\
           &= \frac{m}{m+n}\left[\frac{m^2+mn+m+n-m^2-mn-m^2}{(m+n)(m+n+1)}\right] \\
           &= \frac{m}{m+n}\left[\frac{n}{(m+n)(m+n+1)}\right] \\
    Var(X) &= \frac{mn}{(m+n)^2(m+n+1)}
\end{align*}

Harmonic Mean:
\begin{align*}
    \frac{1}{HM} &= \int \frac{1}{x} f(x) dx \\
    \frac{1}{HM} &= \int \frac{x^{m-2}(1-x)^{n-1}dx}{\beta(m, n)} \\
    \frac{1}{HM} &= \frac{\beta(m-1, n)}{\beta(m, n)} \\
             HM  &= \frac{\beta(m, n)}{\beta(m-1, n)} \\
             HM  &= \frac{(m-1)\Gamma(m-1)\Gamma(m+n-1)}{(m+n-1)\Gamma(m+n-1) \Gamma(m-1)} \\
    HM &= \frac{m-1}{m+n-1}
\end{align*}
    

\paragraph{Raw Moments:}
The value of the \(r^{\text{th}}\) raw moment is
\begin{align*}
    \mu'_r &= \int_0^1 x^r\:f(x)dx \\
           &= \int_0^1 \frac{x^r\;x^{m-1}(1-x)^{n-1}}{\beta(m, n)} \\
           &= \int_0^1 \frac{x^{m+r-1}(1-x)^{n-1}}{\beta(m, n)} \\
           &= \frac{\beta(m+r, n)}{\beta(m, n)} \\
           &= \frac{\Gamma (m+r) \; \Gamma n}{\Gamma (m+n+r)} \frac{\Gamma (m+n)}{\Gamma m \; \Gamma n} \\
    \mu'_r &= \frac{\Gamma (m+r) \Gamma (m+n)}{\Gamma (m+n+r) \Gamma m} \\
    \mu'_r &= \frac{\Pi_{i=0}^{n-1}(m+i)}{\Pi_{i=0}^{n-1}(m+n+i)}
\end{align*}

\newpage
\paragraph*{Type 2(\(\beta_2\))}
If c.r.v \(X\sim \beta_2(m, n)\), then its p.d.f is:
\[
    f(x)= 
\begin{cases}
    \dfrac{1}{\beta(m, n)}\dfrac{x^{m-1}}{(1+x)^{m+n}}, & \text{if } 0 < x < \infty\\ \\
    0,                                     & \text{otherwise}
\end{cases}
\]

Properties:
\[
\beta(m, n) = \int_0^\infty \dfrac{x^{m-1}}{(1+x)^{m+n}} dx
\]
\[
\beta(m, n) = \frac{\Gamma m \; \Gamma n}{\Gamma (m+n)}
\]

\paragraph{Raw Moments:}
The value of the \(r^{\text{th}}\) raw moment is
\begin{align*}
    \mu'_r &= \int_0^\infty x^r\:f(x)dx \\
           &= \int_0^\infty x^r\;\dfrac{1}{\beta(m, n)}\dfrac{x^{m-1}}{(1+x)^{m+n}} \\
           &= \frac{1}{\beta(m, n} \int_0^\infty \frac{x^{m+r-1}}{(1+x)^{n}} \\
           &= \frac{\beta(m+r, n-r)}{\beta(m, n)} \\
           &= \frac{\Gamma (m+r) \; \Gamma (n-r)}{\Gamma (m+n)} \frac{\Gamma (m+n)}{\Gamma m \; \Gamma n} \\
    \mu'_r &= \frac{\Gamma (m+r) \Gamma (n-r)}{\Gamma m \; \Gamma n} \\
    \mu'_r &= \frac{\Pi_{i=0}^{r-1}(m+i)}{\Pi_{i=0}^{r-1}(n-i-1)} 
\end{align*}

\paragraph{Expectation and Variance:}
\begin{align*}
    E(X) &= \mu'_1 \\
         &= \frac{\Gamma (m+1) \Gamma (n-1)}{\Gamma m \; \Gamma n} \\
         &= \frac{m \;\Gamma m \;\Gamma (n-1)}{\Gamma m \;(n-1) \; \Gamma (n-1)} \\
    E(X) &= \frac{m}{n-1} \\ \\
    E(X^2) &= \mu'_2 \\
           &= \frac{\Gamma (m+2) \Gamma (n-2)}{\Gamma m \; \Gamma n} \\
         &= \frac{m(m+1) \Gamma (m) \Gamma (n-2)}{\Gamma m (n-1)(n-2) \Gamma (n-2)} \\ 
    E(X^2) &= \frac{m(m+1)}{(n-1)(n-2)}  \\ \\
    Var(x) &= E(X^2) - [E(X)]^2 \\
           &= \frac{m(m+1)}{(n-1)(n-2)} - \left(\frac{m}{n-1}\right)^2 \\
           &= \frac{m}{n-1} \left( \frac{m+1}{(n-2)} - \frac{m}{(n-1)}\right) \\
           &= \frac{m}{n-1} \left( \frac{(m+1)(n-1) - m(n-2)}{(n-1)(n-2)}\right) \\
    Var(X) &= \frac{m(m+n-1)}{(n-1)^2(n-2)}
\end{align*}

Harmonic Mean:
\begin{align*}
    \frac{1}{HM} &= \int_0^\infty \frac{1}{x} f(x) dx \\
    \frac{1}{HM} &= \int_0^\infty \dfrac{1}{\beta(m, n)}\dfrac{x^{m-2}}{(1+x)^{m+n}} \\
    \frac{1}{HM} &= \frac{\beta(m-1, n+1)}{\beta(m, n)} \\
             HM  &= \frac{\beta(m, n)}{\beta(m-1, n+1)} \\
             HM  &= \frac{\Gamma m \Gamma n}{\Gamma (m+n)} \frac{\Gamma (m+n)}{\Gamma (m-1) \Gamma (n+1)} \\
             HM  &= \frac{(m-1)\Gamma(m-1)\Gamma(n)}{\Gamma(m-1) n \Gamma n} \\
    HM &= \frac{m-1}{n}
\end{align*}

\[
\text{If } X \sim \beta_1(1, 1), X\sim U(0, 1).
\]

\newpage
\subsubsection*{Exponential Distribution:}
If c.r.v \(X\sim \text{exp}(\theta)\), then its p.d.f is:
\[
    f(x)= 
\begin{cases}
    \theta e^{-\theta x}                   & {\theta>0,\; 0<x<\infty} \\
    0,                                     & \text{otherwise}
\end{cases}
\]

\paragraph{Expectation and Variance:}
\begin{align*}
    E(X) &= \int_0^\infty x \cdot f(x) dx \\
         &= \int_0^\infty x\theta e^{-\theta x} \\
         &= \int_0^\infty \frac{u e^{-u}}{\theta} du && \theta x = u, \; dx = \frac{du}{\theta} \\
         &= \frac{1}{\theta} \int_0^\infty u^{2-1} e^{-u} du \\
         &= \frac{\Gamma 2}{\theta} \\
         &= \frac{1!}{\theta} \\
    E(X) &= \frac{1}{\theta} \\ \\
    E(X^2) &= \int_0^1 x^2 \cdot f(x) dx \\
         &= \int_0^\infty x^2\theta e^{-\theta x} \\
         &= \int_0^\infty x\cdot \theta x e^{-\theta x} \\
         &= \int_0^\infty \frac{u}{\theta} \frac{u e^{-u}}{\theta} du && \theta x = u, \; dx = \frac{du}{\theta} \\
         &= \frac{1}{\theta^2} \int_0^\infty u^{3-1} e^{-u} du \\
         &= \frac{\Gamma 3}{\theta^2} \\
         &= \frac{2!}{\theta^2} \\
    E(X^2) &= \frac{2}{\theta^2}  \\ \\
    Var(X) &= E(X^2) - [E(X)]^2 \\
    &= \frac{2}{\theta^2} - \frac{1}{\theta^2} \\
    Var(X) &= \frac{1}{\theta^2}
\end{align*}

\paragraph{Moment Generating Function:}
\[
M_x(t)=(1-\sfrac{t}{\theta})^{-1}
\]

\paragraph{Cumulant Generating Function:}
The value of the \(n^{\text{th}}\) cumulant is \(\dfrac{(n-1)!}{\theta^n}\).

\paragraph{Raw Moments:}
The value of the \(r^{\text{th}}\) raw moment is
\begin{align*}
    \mu'_r &= \int_0^\infty x^r\:f(x)dx \\
           &= \int_0^\infty x^r \cdot \theta e^{-\theta x} dx \\
           &= \int_0^\infty \left(\frac{u}{\theta}\right)^r\cdot\theta e^{-u}\cdot\frac{du}{\theta} && {u=\theta x} \\
           &= \frac{1}{\theta^r} \int_0^\infty u^r e^{-u} du \\
           &= \frac{\Gamma(r+1)}{\theta^r} \\
    \mu'_r &= \frac{r!}{\theta^r}
\end{align*}

\paragraph{Coefficients of Skewness and Kurtosis:}
The coefficients of skewness and kurtosis for the exponential distribution are:
\begin{align*}
    \beta_1 &= \frac{\mu_3^2}{\mu_2^3} \\
            &= \frac{(\sfrac{2}{\theta^3})^2}{(\sfrac{1}{\theta^2})^3} \\
    \beta_1 &= 4 \\ \\
    \gamma_1 &= \sqrt{\beta_1} \\
             &= \sqrt{4} \\
    \gamma_1 &= 2 \\ \\
    \beta_2 &= \frac{\mu_4}{\mu_2^2} \\
            &= \frac{\sfrac{6}{\theta^4}}{(\sfrac{1}{\theta^2})^2} \\
    \beta_2 &= 6 \\ \\
    \gamma_2 &= \beta_2 - 3 \\
             &= 6 - 3 \\
    \gamma_2 &= 3 \\
\end{align*}

\newpage

\subsubsection*{Normal Distribution:}

If c.r.v \(X\sim N\) with mean \(\mu\) and variance \(\sigma^2\), then its p.d.f is:
\[
    f(x)= 
\begin{cases}
    \dfrac{1}{\sigma\sqrt{2\pi}} \; \text{exp} \left[ \dfrac{-1}{2}\left( \dfrac{x-\mu}{\sigma}\right)^2\right] & {-\infty < x < \infty, \;-\infty < \mu < \infty, \;0 < \sigma < \infty} \\
    0,                                     & \text{otherwise}
\end{cases}
\]

It is denoted as \(X\sim N(\mu, \sigma^2) \)

\paragraph*{Properties:}
1) The normal distribution follows a bell-shaped curve. \\
2) In a normal distribution, mean=median=mode. \\
3) The normal distribution is symmetric. \\
4) The scale parameter \(\sigma\) distributes the curve in the following percentage: \\
i) \(\mu \pm \sigma\) contains \(\sim68\%\) data \\
ii) \(\mu \pm 2\sigma\) contains \(\sim95\%\) data \\
iii) \(\mu \pm 3\sigma\) contains \(\sim99.73\%\) data \\
5) \(Z=\frac{x-\mu}{\sigma} \sim\) s.n.d. i.e. \(N(0, 1)\).
\[
P(Z=z) = \frac{1}{\sqrt{2\pi}} \; e^{-\frac{1}{2}z^2}
\]

\paragraph*{Median and Mode:}

Assuming median\(> \mu\),
\begin{align*}
    \implies \int_{-\infty}^\mu f(x)dx + \int_\mu^M f(x)dx &= \frac{1}{2} \\
    \implies 1/2 + \int_\mu^M f(x)dx &= \frac{1}{2} \\
    \implies \int_\mu^M f(x)dx &= 0 \\
    \implies \int_\mu^M \dfrac{1}{\sigma\sqrt{2\pi}} \; e^{-\frac{1}{2}( \frac{x-\mu}{\sigma})^2} &= 0 \\
    \implies \dfrac{1}{\sigma\sqrt{2\pi}} \int_\mu^M e^{-\frac{1}{2\sigma^2}(x-\mu)^2} &= 0 \\
    \implies (x-\mu) \big|_\mu^M = 0 \\
    \implies M - \mu = 0 \\
    \implies M = \mu
\end{align*}

To find mode, we will use the first derivative test, i.e. \(f'(x) = 0\).
\begin{align*}
    log(f(x)) &= -log(\sigma\sqrt{2\pi}) - \frac{(x-\mu)^2}{2\sigma^2} \\
    \text{Differentiating, we get} \\
    \frac{f'(x)}{f(x)} &= -\frac{x-\mu}{\sigma^2} \\
    \implies f(x) (x-\mu) &= 0 \\
    \text{As \(f(x)\) cannot be 0 at the mode}, x-\mu &= 0 \\
    \implies x &= \mu.
\end{align*}
Hence, the second property (mean = mode = median) is proved.

\paragraph*{Moment Generating Function and Cumulant Generating Function:}

\begin{align*}
    M_x(t) &= E(e^{tx}) \\
           &= \int_{-\infty}^\infty e^{tx} \cdot f(x) dx \\
           &= \int_{-\infty}^\infty e^{tx} \cdot \dfrac{1}{\sigma\sqrt{2\pi}} \; e^{-\frac{1}{2}( \frac{x-\mu}{\sigma})^2} dx \\
           &= \dfrac{1}{\sigma\sqrt{2\pi}} \int_{-\infty}^\infty e^{tx-\frac{1}{2}( \frac{x-\mu}{\sigma})^2} dx \\
           \text{Let } z &= \frac{x-\mu}{\sigma} \\ \implies x &= \sigma z + \mu \\ \& \; dx &= \sigma dz \\
           &= \dfrac{1}{\sigma\sqrt{2\pi}} \int_{-\infty}^\infty e^{t(\sigma z + \mu) - \frac{z^2}{2}} \sigma dz\\
           &= \dfrac{1}{\sqrt{2\pi}} \int_{-\infty}^\infty e^{tz\sigma} \cdot e^{t\mu} \cdot e^{- \frac{z^2}{2}} dz \\
           &= \dfrac{e^{t\mu}}{\sqrt{2\pi}} \int_{-\infty}^\infty e^{-\frac{1}{2} (z^2-2tz\sigma)} dz\\
           &= \dfrac{e^{t\mu}}{\sqrt{2\pi}} \int_{-\infty}^\infty e^{-\frac{1}{2} (z^2-2tz\sigma + t^2 \sigma^2 -t^2 \sigma^2)} dz\\
           &= \dfrac{e^{t\mu+\frac{t^2\sigma^2}{2}}}{\sqrt{2\pi}} \int_{-\infty}^\infty e^{-\frac{1}{2} (z-t\sigma)^2} dz\\
           \text{Let } \theta &= z-t\sigma \\ \implies \; dz &= d\theta \\
           &= \dfrac{e^{t\mu+\frac{t^2\sigma^2}{2}}}{\sqrt{2\pi}} \int_{-\infty}^\infty e^{-\frac{\theta^2}{2} d\theta} \\
           &= \dfrac{e^{t\mu+\frac{t^2\sigma^2}{2}}}{\sqrt{2\pi}} \cdot \sqrt{2\pi} \\
    M_x(t) &= e^{t\mu+\frac{t^2\sigma^2}{2}} \\ \\
    K_x(t) &= log(M_x(t)) \\
           &= log(e^{t\mu+\frac{t^2\sigma^2}{2}}) \\
           &= t\mu+\frac{t^2\sigma^2}{2}
\end{align*}

Cumulants:

\[
K_1 = \text{coefficient of } t/1! = \mu
\]

\[
K_2 = \text{coefficient of } t^2/2! = \sigma^2
\]

\[
K_3 = \text{coefficient of } t^3/3! = 0
\]

All further cumulants are \(0\).

\paragraph*{Mean Deviation:}
M.D. from \(\Bar{X}\) is calculated as:
\[
\int_{-\infty}^{\infty} |X-\Bar{X}| = \int_{-\infty}^{\infty} |X-\mu|
\]

\begin{align*}
    E(|X-\mu|) &= \int_{-\infty}^{\infty} |x-\mu| f(x) dx \\
               &= \int_{-\infty}^{\infty} |x-\mu| \dfrac{1}{\sigma\sqrt{2\pi}} \; e^{-\frac{1}{2}( \frac{x-\mu}{\sigma})^2} dx \\
               \text{Let } z &= \frac{x-\mu}{\sigma} \\ \implies x-\mu &= \sigma z \\ \& \; dx &= \sigma dz \\
               &= \dfrac{1}{\sigma\sqrt{2\pi}} \int_{-\infty}^\infty |z\sigma| e^{- \frac{z^2}{2}} \sigma dz\\
               &= \dfrac{\sigma}{\sqrt{2\pi}} \int_{-\infty}^\infty |z| e^{- \frac{z^2}{2}} dz\\
               &= \dfrac{2\sigma}{\sqrt{2\pi}} \int_0^\infty z e^{- \frac{z^2}{2}} dz\\
               \text{Let } t &= \frac{z^2}{2} \\ \implies \; dz &= dt/z \\
               &= \sigma \sqrt{\frac{2}{\pi}} \int_0^\infty z e^{-t} \frac{dt}{z} \\
               &= \sigma \sqrt{\frac{2}{\pi}} \int_0^\infty e^{-t} dt \\
               &= \sigma \sqrt{\frac{2}{\pi}} \; -e^{-t} \Big|_0^\infty \\
               &= \sigma \sqrt{\frac{2}{\pi}} (e^{-0} - e^{-\infty}) \\
    E(|X-\mu|) &= \sigma \sqrt{\frac{2}{\pi}} \approx 0.8 \: \sigma
\end{align*}

\paragraph*{Central Moments of Normal Distribution:}


\textbf{Odd Moments:}
\begin{align*}
    \mu_{2n+1} &= E(X-\mu)^{2n+1} \\
               &= \int_{-\infty}^{\infty} (x-\mu)^{2n+1} \dfrac{1}{\sigma\sqrt{2\pi}} \; e^{-\frac{1}{2}( \frac{x-\mu}{\sigma})^2} dx \\
               &= \dfrac{1}{\sigma\sqrt{2\pi}} \int_{-\infty}^{\infty} (x-\mu)^{2n+1} e^{-\frac{1}{2}(\frac{x-\mu}{\sigma})^2} dx \\
               \text{Let } z &= \frac{x-\mu}{\sigma} \\ \implies x-\mu &= \sigma z \\ \& \; dx &= \sigma dz \\
               &= \frac{1}{\sqrt{2\pi}} \int_{-\infty}^{\infty} (z\sigma)^{2n+1} e^{- \frac{z^2}{2}} dz\\
    \mu_{2n+1} &= \frac{\sigma^{2n+1}}{\sqrt{2\pi}} \int_{-\infty}^{\infty} z^{2n+1} e^{- \frac{z^2}{2}} dz
\end{align*}

As \(z^{2n+1}\) is an odd function, and \(e^{- \frac{z^2}{2}}\) is an even function, its product is an odd function. Integrals of odd functions from \(-\infty\) to \(\infty\) are 0. Hence, the integral given equals 0, and \(\mu_{2n+1} = 0 \; \forall \; n \in \mathbb{N}\).

\textbf{Even Moments:}
\begin{align*}
    \mu_{2n} &= \int_{-\infty}^{\infty} (x-\mu)^{2n} \dfrac{1}{\sigma\sqrt{2\pi}} \;             e^{-\frac{1}{2}( \frac{x-\mu}{\sigma})^2} dx \\
             &= \dfrac{1}{\sigma\sqrt{2\pi}} \int_{-\infty}^{\infty} (x-\mu)^{2n} e^{-\frac{1}{2}(\frac{x-\mu}{\sigma})^2} dx \\
             \text{Let } z &= \frac{x-\mu}{\sigma} \\ \implies x-\mu &= \sigma z \\ \& \; dx &= \sigma dz \\
             &= \frac{1}{\sqrt{2\pi}} \int_{-\infty}^{\infty} (z\sigma)^{2n} e^{- \frac{z^2}{2}} dz\\
             &= \frac{\sigma^{2n}}{\sqrt{2\pi}} \int_{-\infty}^{\infty} z^{2n} e^{- \frac{z^2}{2}} dz\\
             \text{Let } t &= \frac{z^2}{2} \\ \implies \; dz &= dt/\sqrt{2t} \\
             &= \frac{\sigma^{2n}}{\sqrt{2\pi}} \int_{-\infty}^{\infty} (2t)^n e^{-t} \frac{dt}{\sqrt{2t}}\\
             &= \frac{2^n \; \sigma^{2n}}{2\sqrt{\pi}} \int_{-\infty}^{\infty} t^{n-\sfrac{1}{2}} \; e^{-t} dt \\
             &= \frac{2^n \; \sigma^{2n}}{2\sqrt{\pi}} 2 \int_0^{\infty} t^{n-\sfrac{1}{2}} \; e^{-t} dt \\
    \mu_{2n} &= \frac{2^n \; \sigma^{2n}}{\sqrt{\pi}} \Gamma (n + \sfrac{1}{2})
\end{align*}

\textbf{Recursive Relation of Even Moments:}
\begin{align*}
    \mu_{2n} &= \frac{2^n \; \sigma^{2n}}{\sqrt{\pi}} \Gamma (n + \sfrac{1}{2}) \\
    \mu_{2n-2} &= \frac{2^{n-1} \; \sigma^{2n-2}}{\sqrt{\pi}} \Gamma (n - \sfrac{1}{2}) \\
    \implies \frac{\mu_{2n}}{\mu_{2n-2}} &= \frac{2^n \; \sigma^{2n} \Gamma (n + \sfrac{1}{2})}{\sqrt{\pi}} \cdot \frac{\sqrt{\pi}} {2^{n-1} \; \sigma^{2n-2} \Gamma (n - \sfrac{1}{2})} \\
    &= \frac{2\sigma^2 \Gamma (n + \sfrac{1}{2})}{\Gamma (n - \sfrac{1}{2})} \\
    &= \frac{2\sigma^2 (n - \sfrac{1}{2} )\Gamma (n - \sfrac{1}{2})}{\Gamma (n - \sfrac{1}{2})} \\
    \frac{\mu_{2n}}{\mu_{2n-2}} &= 2\sigma^2 (n - \sfrac{1}{2}) \\ \\
    \text{Or, } \\
    \mu_{2n} &= \sigma^2 (2n - 1) \cdot \mu_{2n-2}
\end{align*}

\paragraph*{Addition Property:}
Theorem: If \(X_1, X_2, X_3, \dots, X_n\) are independent \(N(\mu_i, \sigma^2_i)\) then \(\sum_{i=1}^n a_i X_i \sim N(\sum_{i=1}^n a_i\mu_i, \sum_{i=1}^n a_i^2\sigma^2_i)\). \\
Proof:
For Normal Distribution,

\[
M_x(t) = e^{t\mu+\frac{t^2\sigma^2}{2}}
\]

\begin{align*}
    M_{X_1+X_2+\dots+X_n} &= M_{X_1}(t) \cdot M_{X_2}(t) \dots M_{X_n}(t) \\
                          &= e^{t\mu_1+\frac{t^2\sigma_1^2}{2}} \cdot e^{t\mu_2+\frac{t^2\sigma_2^2}{2}} \cdot \dots \cdot e^{t\mu_n+\frac{t^2\sigma_n^2}{2}} \\
    M_{\sum_{i=1}^n X_i} &= e^{t(\mu_1+\mu_2+\dots+\mu_n) + \frac{t^2}{2}(\sigma_1^2 + \sigma_2^2 +\dots+ \sigma_n^2)}
\end{align*}

By Uniqueness Theorem of m.g.f., the m.g.f. of any linear combination of n.r.v's follows normal distribution with mean \(\sum_{i=1}^n a_i\mu_i\) and variance \(\sum_{i=1}^n a_i^2\sigma_i^2)\).

\paragraph*{Skewness and Kurtosis:}
We know the first four central moments of the normal distribution are:
\begin{align*}
    \mu_1 &= 0 \\
    \mu_2 &= \sigma^2 \\
    \mu_3 &= 0 \\
    \mu_4 &= K_4 + K_2^2 \\
          &= 3\sigma^4
\end{align*}

This gives us values of skewness and kurtosis:
\begin{align*}
    \beta_1 &= \frac{\mu_3^2}{\mu_2^3} \\
            &= \frac{0}{\sigma^6} \\
            &= 0 \\ \\
    \beta_2 &= \frac{\mu^4}{\mu_2^2} \\
            &= \frac{3\sigma^4}{\sigma^4} \\
            &= 3
\end{align*}

As \(\beta_1=0\) and \(\beta_2=3\), we can definitively say that the normal distribution is always symmetric, irrespective of its parameters.


\subsubsection*{Log-Normal Distribution:}
If \(Y=logX \sim N(\mu, \sigma^2), X\) follows lognormal distribution. 

\begin{align*}
    F_X(X) &= f(X\leq x) \\
           &= f(logX\leq logx) \\
           &= f(Y\leq logx) \\
           &= \int_{-\infty}^{logx} \frac{1}{\sigma\sqrt{2\pi}}\text{exp}\left(-\frac{1}{2} \left(\frac{y-\mu}{\sigma}\right)^2\right)dy
\end{align*}

\begin{align*}
    \mu'_r &= E(x^r) \\
           &= E(e^{yr}) \\
           &= \text{exp}\left( \mu r + \frac{\sigma^2 r^2}{2} \right)
\end{align*}

Hence, 
\begin{align*}
    E(X)    &= \mu'_1 \\
            &= \text{exp}\left( \mu + \frac{\sigma^2}{2} \right) \\\\
    E(X^2)  &= \mu'_2 \\
            &= \text{exp}[2(\mu + \sigma^2)] \\\\
    Var(X)  &= \mu'_2 - \mu_1'^2 \\
            &= \text{exp}[2(\mu + \sigma^2)] - \text{exp}\left( \mu + \frac{\sigma^2}{2} \right)^2
\end{align*}

\subsubsection*{Cauchy Distribution:}
If c.r.v \(X\sim C\), then its p.d.f is:
\[
    f(x)= 
\begin{cases}
    \dfrac{1}{\pi} \dfrac{1}{1+x^2}        & {;-\infty<x<\infty} \\\\
    0,                                     & ;\text{otherwise}
\end{cases}
\]

\[
X = \frac{Y-\mu}{\sigma}
\]
Then, if Y \(\sim C(\lambda, \mu)\), \(X\sim C(1, 0)\).
And, \(X\) is the standard Cauchy distribution.

\[
G_Y(Y) = \frac{\lambda}{\pi(\lambda^2+(Y-\mu)^2)}
\]

Hence,
\begin{align*}
    E(Y) &= \int_{-\infty}^{\infty} y \; g(y) dy \\
         &= \frac{\lambda}{\pi} \int_{-\infty}^{\infty} \frac{y}{(\lambda^2+(y-\mu)^2)} \\
         &= \frac{\mu\lambda}{\pi} \int_{-\infty}^{\infty} \frac{dy}{(\lambda^2+(y-\mu)^2)} + \frac{\lambda}{\pi} \int_{-\infty}^{\infty} \frac{y-\mu}{(\lambda^2+(y-\mu)^2)}dy \\
         &= \mu +  \frac{\lambda}{\pi} \int_{-\infty}^{\infty} \frac{z}{(\lambda^2+z^2)}dz
\end{align*}
This integral is undefined, hence the first moment i.e. the mean does not exist.



\newpage
\paragraph{Bivariate Distributions:}
If \(X\) and \(Y\) are c.r.v.s, then \(F_{XY}(X, Y)\) is the joint p.d.f. iff
\[
0 < f_{XY}(x, y) < 1
\]

If \(f(X, Y)\) is the joint density function then,
\[
\int_X \int_Y f(X, Y) \; dy \; dx = 1
\]

Marginal p.d.f's are:
\[
f_X(x) = \int_Y f(X, Y) \; dy,
\]
\[
f_Y(y) = \int_X f(X, Y) \; dx
\]

Conditional p.d.f's are:
\[
f(Y|X=x) = \frac{f(X, Y)}{f(X=x)}
\]

\[
P(x_1 < X < x_2, y_1 < Y < y_2) = F(x_2, y_2) - f(x_1, y_1)
\]

If \(X\) and \(Y\) are independent, 
\[
f_{XY}(x, y) = f_X(x) \cdot f_Y(y) \; \forall \; x, y
\]


\subsubsection*{Bivariate Normal Distribution:}
If \(X\) \& \(Y\) are jointly distributed with bivariate normal distribution with parameters \((\mu_1, \mu_2, \sigma_1^2, \sigma_2^2, \rho)\), then their j.p.d.f. is:

\[
f(X, Y) = \frac{1}{2\pi\sigma_1\sigma_2\sqrt{1-\rho^2}} \text{exp} \left\{ -\frac{1}{2(1-\rho^2)} \left[ \left( \frac{x-\mu_1}{\sigma_1} \right)^2 - \frac{2\rho(x-\mu_1)(y-\mu_2)}{\sigma_1\sigma_2} + \left( \frac{y-\mu_2}{\sigma_2} \right)^2 \right]\right\}
\]

The j.p.d.f. of s.n.b.d. is:
\[
f(X, Y) = \frac{1}{2\pi\sqrt{1-\rho^2}} \text{exp} \left\{ -\frac{1}{2(1-\rho^2)} \left[ x^2 - 2\rho x y + y^2 \right]\right\}
\]

If \(X\) \& \(Y\) are jointly distributed with bivariate normal distribution with joint probability distribution function \(f(X, Y)\), then conditional probability distribution \(P(X|Y)\) is given as:

\[
\frac{f(X, Y)}{f(Y)} = \frac{1}{\sigma_1 \sqrt{2\pi} \sqrt{1-\rho^2}} \text{exp} \left\{ -\frac{1}{2\sigma_1^2(1-\rho^2)} \left[ x - \left( \mu_1 + \rho \frac{\sigma_1}{\sigma_2} (y-\mu_2) \right) \right]^2 \right\}
\]

Then, 
\[
E(X|Y) = \mu_1 + \rho \frac{\sigma_1}{\sigma_2} (y-\mu_2) 
\]
And,
\[
Var(X|Y) = \sigma_1^2 (1-\rho^2)
\]
\[
S.D.(X|Y) = \sigma_1 \sqrt{1-\rho^2}
\]




%----------------------------------------------------------------------------------------

\end{document}