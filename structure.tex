%%%%%%%%%%%%%%%%%%%%%%%%%%%%%%%%%%%%%%%%%
% Article Notes
% Structure Specification File
% Version 1.0 (1/10/15)
%
% This file has been downloaded from:
% http://www.LaTeXTemplates.com
%
% Authors:
% Vel (vel@latextemplates.com)
% Christopher Eliot (christopher.eliot@hofstra.edu)
% Anthony Dardis (anthony.dardis@hofstra.edu)
%
% License:
% CC BY-NC-SA 3.0 (http://creativecommons.org/licenses/by-nc-sa/3.0/)
%
%%%%%%%%%%%%%%%%%%%%%%%%%%%%%%%%%%%%%%%%%

%----------------------------------------------------------------------------------------
%	REQUIRED PACKAGES
%----------------------------------------------------------------------------------------

\usepackage[includeheadfoot,columnsep=2cm, left=1in, right=1in, top=.5in, bottom=.5in]{geometry} % Margins

\usepackage[T1]{fontenc} % For international characters
\usepackage{XCharter} % XCharter as the main font

\usepackage[english]{babel} % Use english by default

% maths packages

\usepackage{amsmath}
\usepackage{xfrac}
\usepackage{multirow}
\usepackage{float}
\usepackage{xcolor}
\usepackage{enumerate}
\usepackage{listings}

%----------------------------------------------------------------------------------------
%	CUSTOM COMMANDS
%----------------------------------------------------------------------------------------

\newcommand{\articletitle}[1]{\renewcommand{\articletitle}{#1}} % Define a command for storing the article title

\newcommand{\datenotesstarted}[1]{\renewcommand{\datenotesstarted}{#1}} % Define a command to store the date when notes were first made
\newcommand{\docdate}[1]{\renewcommand{\docdate}{#1}} % Define a command to store the date line in the title

\newcommand{\docauthor}[1]{\renewcommand{\docauthor}{#1}} % Define a command for storing the article notes author

% Define a command for the structure of the document title
\newcommand{\printtitle}{
\begin{center}
\topskip0pt
\vspace*{\fill}
\textbf{\Huge{The Notes Project}}

\vspace{2ex}

\textbf{\huge{\articletitle}}

\vspace{1ex}

\Large\docdate \\
\vspace{1ex}
\Large\docauthor \\
\vspace*{\fill}

\end{center}
}

% \renewcommand{\theenumii}{\alph{enumii}.}

\usepackage{fancyhdr}

\newcommand{\ls}[2]{\sum_{#1=1}^{#2}}

\newcommand{\comment}[1]{%
  \text{\phantom{(#1)}} \tag{#1}
}

%----------------------------------------------------------------------------------------
%	STRUCTURE MODIFICATIONS
%----------------------------------------------------------------------------------------

\setlength{\parskip}{3pt} % Slightly increase spacing between paragraphs

% Uncomment to center section titles
\usepackage{sectsty}
\sectionfont{\centering}

% Uncomment for Roman numerals for section numbers
\renewcommand\thesection{\Roman{section}}


\usepackage{pifont}

\usepackage{tikz, pgfplots}
\pgfplotsset{compat=1.18}
\usetikzlibrary{positioning}

\usepackage{gensymb}


% ------------------------
%        PIE CHART
% ------------------------

\newcommand{\pieslice}[6][black!10]{
%%% Usage: \pieslice[color]{total}{start angle}{end angle}{data value}{label}
  % calculate start and end points of arc
  \pgfmathparse{#3/#2*360}
  \let\a\pgfmathresult
  \pgfmathparse{#4/#2*360}
  \let\b\pgfmathresult

  % calculate mid angle of arc
  \pgfmathparse{0.5*\a+0.5*\b}
  \let\midangle\pgfmathresult

  % draw slice
  \draw[fill=#1] (0,0) -- (\a:1) arc (\a:\b:1) -- cycle;

  % outer label
  \node[label=\midangle:{\small#6}] at (\midangle:1) {};

  % inner label
  \pgfmathparse{min((\b-\a-10)/110*(-0.3),0)}
  \let\temp\pgfmathresult
  \pgfmathparse{max(\temp,-0.5) + 0.8}
  \let\innerpos\pgfmathresult
  \pgfmathparse{(\b-\a)/3.6} % convert slice size to percentage
  \let\percentage\pgfmathresult
  \node at (\midangle:\innerpos) {\small\pgfmathprintnumber[fixed,precision=1]{\percentage}\%};
}

\newcommand{\pie}[2][{{"black!10"}}]{
%%% Usage: \pie[{colour palette array}]{{label/value array}}
  % init colour palette
  \pgfmathparse{dim(#1)} % find N of array
  \let\paletteDim\pgfmathresult
  \newcounter{colourIndex}

  % get total for dividing pie into sectors
  \newcounter{total}
  \foreach \val/\name in #2 {
    \addtocounter{total}{\val}
  }

  \newcounter{a}
  \newcounter{b}
  \foreach \val/\name in #2 {
    \setcounter{a}{\value{b}}
    \addtocounter{b}{\val}

    % get colour from palette
    \pgfmathparse{#1[\thecolourIndex]}
    \let\colour\pgfmathresult

    \pieslice[\colour]{\thetotal}{\thea}{\theb}{\val}{\name}

    % increment colour palette
    \stepcounter{colourIndex}
    \ifnum \thecolourIndex=\paletteDim \setcounter{colourIndex}{0}\fi
  }
}

\def\palette{{"blue!60","cyan!50","yellow!50","orange!60","red!60",
    "teal!50","brown!50!black!50","purple!50","lime!50!black!30"}}
    
    
% ---------------------------
%           TALLY
% ---------------------------

\newcount\tmpnum
\def\tallymarks#1{\leavevmode \lower1bp\vbox to9bp{}%
   \tmpnum=#1
   \loop \ifnum\tmpnum<5 \kern1bp \tallynum\tmpnum \else \tallyV \fi
         \advance\tmpnum by-5
         \ifnum\tmpnum>0 \repeat
}
\def\tallynum#1{\bgroup\tmpnum=#1\relax
   \loop \ifnum\tmpnum>0
         \kern1bp \tallyI \kern1bp
         \advance\tmpnum by-1
         \repeat
   \egroup
}
\def\tallyI{\pdfliteral{q .5 w 0 -1 m 0 8 l S Q}}
\def\tallyV{\kern1bp\pdfliteral{q .5 w -1 0 m 9 7 l S Q}\tallynum4\kern1bp }

